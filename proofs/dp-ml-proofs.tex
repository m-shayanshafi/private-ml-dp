\documentclass{article}
\usepackage[utf8]{inputenc}
\usepackage[english]{babel}
% \usepackage	{bibtex}

\usepackage{blindtext}
\usepackage{amssymb}

\usepackage{amsthm}
\usepackage{amsmath}
\usepackage{amsmath}
%% The following is a directive for TeXShop to indicate the main file
%%!TEX root = dp-ml-proofs.tex
\usepackage{ifthen}
\usepackage[normalem]{ulem} % for \sout
\usepackage{xcolor}
\usepackage{amssymb}
\newboolean{showedits}
\setboolean{showedits}{true} % toggle to show or hide edits
%\setboolean{showedits}{false} 
\ifthenelse{\boolean{showedits}}
{
	\newcommand{\ugh}[1]{\textcolor{red}{\uwave{#1}}} % please rephrase
	\newcommand{\ins}[1]{\textcolor{blue}{\uline{#1}}} % please insert
	\newcommand{\del}[1]{\textcolor{red}{\sout{#1}}} % please delete
	\newcommand{\chg}[2]{\textcolor{red}{\sout{#1}}{\ra}\textcolor{blue}{\uline{#2}}} % please change
}{
	\newcommand{\ugh}[1]{#1} % please rephrase
	\newcommand{\ins}[1]{#1} % please insert
	\newcommand{\del}[1]{} % please delete
	\newcommand{\chg}[2]{#2}
}

\newboolean{showcomments}
\setboolean{showcomments}{true}
%\setboolean{showcomments}{false}
\newcommand{\id}[1]{$-$Id: scgPaper.tex 32478 2010-04-29 09:11:32Z oscar $-$}
\newcommand{\yellowbox}[1]{\fcolorbox{gray}{yellow}{\bfseries\sffamily\scriptsize#1}}
\newcommand{\triangles}[1]{{\sf\small$\blacktriangleright$\textit{#1}$\blacktriangleleft$}}
\ifthenelse{\boolean{showcomments}}
%{\newcommand{\nb}[2]{{\yellowbox{#1}\triangles{#2}}}
{\newcommand{\nbc}[3]{
 {\colorbox{#3}{\bfseries\sffamily\scriptsize\textcolor{white}{#1}}}
 {\textcolor{#3}{\sf\small$\blacktriangleright$\textit{#2}$\blacktriangleleft$}}}
 \newcommand{\version}{\emph{\scriptsize\id}}}
{\newcommand{\nbc}[3]{}
 \renewcommand{\ugh}[1]{#1} % please rephrase
 \renewcommand{\ins}[1]{#1} % please insert
 \renewcommand{\del}[1]{} % please delete
 \renewcommand{\chg}[2]{#2} % please change
 \newcommand{\version}{}}
\newcommand{\nb}[2]{\nbc{#1}{#2}{orange}}


\definecolor{ibcolor}{rgb}{0.4,0.6,0.2}
\definecolor{cfcolor}{rgb}{0.2,0.6,0.6}
\definecolor{smcolor}{rgb}{0.8,0.5,0.3}
\definecolor{mlcolor}{rgb}{0.0,0.8,0.2}

\newcommand\iv[1]{\nbc{IB}{#1}{ibcolor}}
\newcommand\sh[1]{\nbc{MS}{#1}{smcolor}}
\newcommand\todo[1]{\nbc{TODO}{#1}{todocolor}}
\newcommand\ml[1]{\nbc{ML}{#1}{cfcolor}}


\usepackage{wasysym}
\newcommand\yesml[1]{\nbc{ML {\textcolor{yellow}\sun}}{#1}{mircolor}}

\definecolor{todocolor}{rgb}{0.9,0.1,0.1}



%\newtheorem{theorem}{Theorem}
\newtheorem{theorem}{Theorem}[section]
\newtheorem{corollary}{Corollary}[theorem]
\newtheorem{lemma}[theorem]{Lemma}

\theoremstyle{remark}
\newtheorem*{remark}{Remark}

\theoremstyle{definition}
\newtheorem{definition}{Definition}[section]

\title{DP-ML Proofs}
% \author{}
% \date{}


\renewcommand\qedsymbol{$\blacksquare$}
% %% The following is a directive for TeXShop to indicate the main file
%%!TEX root = dp-ml-proofs.tex
\usepackage{ifthen}
\usepackage[normalem]{ulem} % for \sout
\usepackage{xcolor}
\usepackage{amssymb}
\newboolean{showedits}
\setboolean{showedits}{true} % toggle to show or hide edits
%\setboolean{showedits}{false} 
\ifthenelse{\boolean{showedits}}
{
	\newcommand{\ugh}[1]{\textcolor{red}{\uwave{#1}}} % please rephrase
	\newcommand{\ins}[1]{\textcolor{blue}{\uline{#1}}} % please insert
	\newcommand{\del}[1]{\textcolor{red}{\sout{#1}}} % please delete
	\newcommand{\chg}[2]{\textcolor{red}{\sout{#1}}{\ra}\textcolor{blue}{\uline{#2}}} % please change
}{
	\newcommand{\ugh}[1]{#1} % please rephrase
	\newcommand{\ins}[1]{#1} % please insert
	\newcommand{\del}[1]{} % please delete
	\newcommand{\chg}[2]{#2}
}

\newboolean{showcomments}
\setboolean{showcomments}{true}
%\setboolean{showcomments}{false}
\newcommand{\id}[1]{$-$Id: scgPaper.tex 32478 2010-04-29 09:11:32Z oscar $-$}
\newcommand{\yellowbox}[1]{\fcolorbox{gray}{yellow}{\bfseries\sffamily\scriptsize#1}}
\newcommand{\triangles}[1]{{\sf\small$\blacktriangleright$\textit{#1}$\blacktriangleleft$}}
\ifthenelse{\boolean{showcomments}}
%{\newcommand{\nb}[2]{{\yellowbox{#1}\triangles{#2}}}
{\newcommand{\nbc}[3]{
 {\colorbox{#3}{\bfseries\sffamily\scriptsize\textcolor{white}{#1}}}
 {\textcolor{#3}{\sf\small$\blacktriangleright$\textit{#2}$\blacktriangleleft$}}}
 \newcommand{\version}{\emph{\scriptsize\id}}}
{\newcommand{\nbc}[3]{}
 \renewcommand{\ugh}[1]{#1} % please rephrase
 \renewcommand{\ins}[1]{#1} % please insert
 \renewcommand{\del}[1]{} % please delete
 \renewcommand{\chg}[2]{#2} % please change
 \newcommand{\version}{}}
\newcommand{\nb}[2]{\nbc{#1}{#2}{orange}}


\definecolor{ibcolor}{rgb}{0.4,0.6,0.2}
\definecolor{cfcolor}{rgb}{0.2,0.6,0.6}
\definecolor{smcolor}{rgb}{0.8,0.5,0.3}
\definecolor{mlcolor}{rgb}{0.0,0.8,0.2}

\newcommand\iv[1]{\nbc{IB}{#1}{ibcolor}}
\newcommand\sh[1]{\nbc{MS}{#1}{smcolor}}
\newcommand\todo[1]{\nbc{TODO}{#1}{todocolor}}
\newcommand\ml[1]{\nbc{ML}{#1}{cfcolor}}


\usepackage{wasysym}
\newcommand\yesml[1]{\nbc{ML {\textcolor{yellow}\sun}}{#1}{mircolor}}

\definecolor{todocolor}{rgb}{0.9,0.1,0.1}




\begin{document}


\maketitle

% To leave comments, use the following commands
% Ivan => \iv{This is a sample comment}
% Mathias => \ml{This is a sample comment}
% Shayan => \sh{This is a sample comment}


\section{Privacy Loss when adding $\epsilon_{1} + \epsilon_{2}$ noise}

Let:

$A(x) => (\epsilon_{1}, \delta)$ differentially private mechanism by adding noise from $N(0, \sigma_{1}^{2})$
$B(x) => (\epsilon_{2}, \delta)$ differentially private mechanism by adding noise from $N(0, \sigma_{2}^{2})$

\begin{theorem}
If C(x) is a dp-mechanism that adds the sum of noise sampled from $N(0, \sigma_{1}^{2})$ and $N(0, \sigma_{2}^{2})$, then C(x) is $\sqrt(\frac{\epsilon_{1}^{2}*\epsilon_{2}^{2}}{\epsilon_{1}^{2} + \epsilon_{2}^{2}})$ differentially private.

\end{theorem}

\begin{proof}

We know that the sum of two normal distributed random variable is also normal $=> N(0, \sigma_{1}^{2}) + N(0, \sigma_{2}^{2}) = N(0, \sigma_{1}^{2} + \sigma_{2}^{2})$	

Therefore, summing up noise from two randomly distributed variables  is equivalent to sampling noise from $N(0, \sigma_{1}^{2} + \sigma_{2}^{2} = \sigma_{3}^{2})$  	

From \cite{Dwork:2014} it follows that, if $\sigma$ is equivalent to

\begin{align*}
	\frac{s}{\epsilon}\sqrt(2ln\frac{1.25}{\delta})
\end{align*}

then a step is $(\epsilon,\delta)$ differentially private.

We know:
	
\begin{gather*}
	\sigma_{3}^{2} = \sigma_{1}^{2} + \sigma_{2}^{2}\\ 
	\sigma_{3}^{2} = \frac{s^{2}}{\epsilon_1^{2}}(2ln\frac{1.25}{\delta}) + \frac{s^{2}}{\epsilon_1^{2}}(2ln\frac{1.25}{\delta}) \\
	\sigma_{3}^{2} = (2ln\frac{1.25}{\delta})(\frac{s^{2}}{\epsilon_{1}^{2}} + \frac{s^{2}}{\epsilon_{1}^{2}})\\ 
	\frac{s^{2}}{\epsilon_{3}^{2}}(2ln\frac{1.25}{\delta}) = (2ln\frac{1.25}{\delta})(\frac{s^{2}}{\epsilon_{1}^{2}} + \frac{s^{2}}{\epsilon_{2}^{2}})\\
	\frac{s^{2}}{\epsilon_{3}^{2}}(2ln\frac{1.25}{\delta}) = (2ln\frac{1.25}{\delta})(\frac{s^{2}}{\epsilon_{1}^{2}} + \frac{s^{2}}{\epsilon_{2}^{2}})\\ 
	\frac{s^{2}}{\epsilon_{3}^{2}}(2ln\frac{1.25}{\delta}) = (2ln\frac{1.25}{\delta})(\frac{s^{2}}{\epsilon_{1}^{2}} + \frac{s^{2}}{\epsilon_{2}^{2}})\\ 	
\end{gather*}

\begin{gather*}
	\frac{1}{\epsilon_{3}^{2}} = \frac{1}{\epsilon_{1}^{2}} + \frac{1}{\epsilon_{2}^{2}}\\
	\frac{1}{\epsilon_{3}^{2}} = \frac{\epsilon_{1}^{2} + \epsilon_{2}^{2}}{\epsilon_{1}^{2} * \epsilon_{2}^{2}}\\
	\epsilon_{3}^{2} = \frac{\epsilon_{1}^{2} * \epsilon_{2}^{2}}{\epsilon_{1}^{2} + \epsilon_{2}^{2}}\\
	\epsilon_{3} = \sqrt(\frac{\epsilon_{1}^{2} * \epsilon_{2}^{2}}{\epsilon_{1}^{2} + \epsilon_{2}^{2}})
\end{gather*}

\end{proof}

Here are a few examples for the resulting epsilon value  when you add two noise vectors each satisfying $(\epsilon_{1},\delta)$ and $(\epsilon_{2},\delta)$ respectively:

	\begin{center}
	 \begin{tabular}{||c c c||} 
	 \hline
	 $\epsilon_{1}$ & $\epsilon_{2}$ & $\epsilon_{3}$ \\  
	 \hline\hline
	 0.5 & 0.75 & 0.416 \\
	 \hline
	 1.0 & 0.5 & 0.447 \\
	 \hline
	 2.0 & 1.0 & 0.894 \\
	 \hline
	 2.0 & 0.5 & 0.485 \\
	\end{tabular}
	\end{center}

\sh{Not sure if the above result can be used for $\epsilon > 1$ values since $\frac{s}{\epsilon}\sqrt(2ln\frac{1.25}{\delta})$ might only applicable for cases where both $\epsilon_{1} < 1$ and $\epsilon_{2} < 1$ \cite{Dwork:2014}}

\section{Next Steps/TODOs}

Here are some potential next steps/proofs for designing a private Federated Learning system with an untrusted aggregator:

 
1. \emph{How much privacy gain do you get when only observing only the noisy aggregate compared to observing all the individual updates?}

	Some Ideas $=>$ From the secure aggregation paper (Appendix A) ~\cite{Bonawitz:2017}, we know that we can get the same $\epsilon$- guarantee with secure aggregation by sampling from $N(0,\sigma/\sqrt(n))$ compared to sampling from $N(0,\sigma)$ with no secure aggregation.

	By using the same argument, if A(x) satisfies $(\epsilon,\delta)$- differential privacy when the adversary observes each update  protected by noise sampled from $N(0, \sigma)$, then with the adversary observing only the aggregate of the noisy updates, it becomes equivalent to $(\epsilon,\delta)$ guarantee by sampling from $N(0, \sigma*\sqrt(n))$ . \sh{??}

	However, a key assumption to get the gain above would be a shift to computational differential privacy~\cite{Mironov:2009}. Not sure currently that if we make this assumption how it would affect privacy calculations? 

	\sh{With computation differential privacy can I still use $\frac{s}{\epsilon}\sqrt(2ln\frac{1.25}{\delta})$to get $(\epsilon,\delta)$ differential privacy here?}

2. \emph{How would the privacy loss compose over N rounds such that an adversary observes the individual dp-updates in m out of N rounds and the aggregate in others?}  

	Would it be possible to use one composition theorem out of the box for this?

	Some potential compositions that can be used:
	\begin{enumerate}
		\item Advanced composition theorem ~\cite{Vadhan:2017} $=>O(\sqrt(klog(\frac{1}{\delta'})).\epsilon, k*\delta + \delta')$ $(k < 1/\epsilon^{2})$
		\item Moments Accountant ~\cite{Abadi:2016} $=>$ $O(q*\epsilon*sqrt(T))$ 
		\item Amplification by sampling $O(q*\epsilon, q*\delta)$ ~\cite{Abadi:2016} 

	\end{enumerate}

{\footnotesize \bibliographystyle{acm}
\bibliography{dp-ml-proofs}}

\end{document}